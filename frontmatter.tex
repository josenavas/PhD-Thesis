%
%
% UCSD Doctoral Dissertation Template
% -----------------------------------
% http:\\ucsd-thesis.googlecode.com
%
%


%% REQUIRED FIELDS -- Replace with the values appropriate to you

% No symbols, formulas, superscripts, or Greek letters are allowed
% in your title.
\title{Making pipelines and large processing microbial community studies available to any user, any time, any place}

\author{Jose Antonio Navas Molina}
\degreeyear{2018}

% Master's Degree theses will NOT be formatted properly with this file.
\degreetitle{Doctor of Philosophy}

\field{Computer Science}
\chair{Professor Rob Knight}
% Uncomment the next line iff you have a Co-Chair
% \cochair{Professor Cochair Semimaster}
%
% Or, uncomment the next line iff you have two equal Co-Chairs.
%\cochairs{Professor Chair Masterish}{Professor Chair Masterish}

%  The rest of the committee members  must be alphabetized by last name.
\othermembers{
Professor Nuno Bandeira\\
Professor Vineet Bafna\\
Professor Pieter Dorrestein\\
Professor Larry Smarr\\
}
\numberofmembers{5} % |chair| + |cochair| + |othermembers|


%% START THE FRONTMATTER
%
\begin{frontmatter}

%% TITLE PAGES
%
%  This command generates the title, copyright, and signature pages.
%
\makefrontmatter

%% DEDICATION
%
%  You have three choices here:
%    1. Use the ``dedication'' environment.
%       Put in the text you want, and everything will be formated for
%       you. You'll get a perfectly respectable dedication page.
%
%
%    2. Use the ``mydedication'' environment.  If you don't like the
%       formatting of option 1, use this environment and format things
%       however you wish.
%
%    3. If you don't want a dedication, it's not required.
%
%
% \begin{dedication}
% 	TODO: dedication
% \end{dedication}

% You are responsible for formatting here.
%\begin{mydedication}
%  \vspace{1in}
%  \begin{flushleft}
%    To me.
%  \end{flushleft}
%
%   \vspace{2in}
%   \begin{center}
%     And you.
%   \end{center}
%
%  \vspace{2in}
%  \begin{flushright}
%    Which equals us.
%  \end{flushright}
%\end{mydedication}



%% EPIGRAPH
%
%  The same choices that applied to the dedication apply here.
%

% The style file will position the text for you.
\begin{epigraph}
  \emph{First, solve the problem. Then, write the code}\\
  ---John Johnson
\end{epigraph}

%% SETUP THE TABLE OF CONTENTS
%
\tableofcontents

%%
%% This block was needed to re-format the title of the glossary to match the
%% headings of the list of figures and list of tables.
%%
%% start hack:
\renewcommand{\glossarysection}[2][]{
\newpage
\noindent
\centerline{LIST OF ABBREVIATIONS}
}
%% end hack

\printglossary[title=List of Abbreviations,toctitle=List of Abbreviations,nonumberlist ]
\listoffigures  % Uncomment if you have any figures
\listoftables   % Uncomment if you have any tables

%% ACKNOWLEDGEMENTS
%
%  While technically optional, you probably have someone to thank.
%  Also, a paragraph acknowledging all coauthors and publishers (if
%  you have any) is required in the acknowledgements page and as the
%  last paragraph of text at the end of each respective chapter. See
%  the OGS Formatting Manual for more information.
%
\begin{acknowledgements}

	% The contents of this type of acknowledgements has to be the same that
	% the paragraph that appears in the thesis contents.
    Section~\ref{section_bigdata}, in full, reproduces the material as it
    appears in ``The microbiome and big data''. J. A. Navas-Molina, E. R. Hyde,
    J. G. Sanders and R. Knight. \emph{Current Opinion in Systems Biology},
    2017, DOI: 10.1016/j.coisb.2017.07.003.

	Section~\ref{section_book}, in full, reproduces the material as it
    appears in ``Advancing our understanding of the human microbiome using QIIME''.
	J. A. Navas-Molina, J. M. Peralta-Sanchez, A. Gonzalez, P. J. McMurdie,
	Y. Vazquez-Baeza, Z. Xu, L. K. Ursell, C. Lauber, H. Zhou, S. J. Song,
	J. Huntley, G. L. Ackermann, D. Berk-Lyons, S. Holmes, J. G. Caporaso and R.
	Knight. \emph{Methods in Enzymology}, 2013, DOI: 10.1016/B978-0-12-407863-5.00019-8.

    Section~\ref{subsection_openref} has been adapted from the original publication in
    ``Subsampled open-reference clustering creates consistent, comprehensive OTU
    definitions and scales to billions of sequences''. J. R. Rideout, Y. He,
    J. A. Navas-Molina, W.A. Walters, L. K. Ursell, S. M. Gibbons, J. Chase,
    D. McDonald, A. Gonzalez, A. Robbins-Pianka, J. C. Clemente, J. A. Gilbert,
    S. M. Huse, H. W. Zhou and R. Knight \emph{PeerJ}, 2014, DOI: 10.7717/peerj.545.

    Section~\ref{subsection_soa} has been adapted from the original publication in
    ``Open-source sequence clustering methods improve the State of the Art''.
    E. Kopylova, J. A. Navas-Molina, C. Mercier, Z. Z. Xu, F. Mahe, Y. He, H. Zhou,
    T. Rognes, J. G. Caporaso, R. Knight \emph{mSystems}, 2016, DOI: 10.1128/mSystems.00003-15

    Section~\ref{subsection_deblur} has been adapted from the original publication in
    ``Deblur rapidly resolves single-nucleotide community sequence patterns''.
    A. Amir, D. McDonald, J. A. Navas-Molina, E. Kopylova, J. T. Morton, Z. Z. Xu,
    E. P. Kightley, L. R. Thompson, E. R. Hyde, A. Gonzalez, R. Knight \emph{mSystems},
    2017, DOI: 10.1128/mSystems.00191-16

    Section~\ref{subsection_komodo} has been adapted from the original publication in
    ``The oral and skin microbiomes of captive Komodo dragons are significantly shared
    with their habitat''. E.R. Hyde, J. A. Navas-Molina, S. J. Song, J. G. Kueneman,
    G. Ackermann, C. Cardona, G. Humphrey, D. Boyer, T. Weaver, J. R. Mendelson,
    V. J. McKenzie, J. A. Gilbert, R. Knight \emph{mSystems}, 2016. DOI: 10.1128/mSystems.00046-16

    Section~\ref{subsection_emp} has been adapted from the original publication in
    ``A communal catalogue reveals Earth's multiscale microbial diversity''.
    \emph{Nature} L. R. Thompson, J. G. Sanders, D. McDonald, A. Amir,
    J. Ladau, K. J. Locey, R. J. Prill, A. Tripathi, S. M. Gibbons, G. Ackermann,
    J. A. Navas-Molina, S. Janssen, E. Kopylova, Y. Vazquez-Baeza, A. Gonzalez,
    J. T. Morton, S. Mirarab, Z. Z. Xu, L. Jiang, M. F. Haroon, J. Kanbar, Q. Zhu,
    S. J. Song, T. Kosciolek, N. A. Bokulich, J. Lefler, C. J. Brislawn, G. Humphrey,
    S. M. Owens, J. Hampton-Marcell, D. Berg-Lyons, V. McKenzie, N. Fierer, J. A. Fuhrman,
    A. Clauset, R. L. Stevens, A. Shade, K. S. Pollard, K. D. Goodwin, J. K. Jansson,
    J. A. Gilbert, R. Knight, The Earth Microbiome Project Consortium, 2017. DOI: 10.0.4.14/nature24621

    Section~\ref{subsection_ag}, in part, has been submitted for publication of the
    material as it may appear in Science, 2018, D. McDonald, E. R. Hyde, J. W. Debelius,
    J. T. Morton, A. Gonzalez, G. Ackermann, A. A. Aksenov, B. Behsaz, C. Brennan,
    Y. Chen, L. DeRight Goldasich, P. C. Dorrestein, R. R. Dunn, A. K. Fahimipour,
    J. Gaffney, J. A Gilbert, G. Gogul, J. L. Green, P. Hugenholtz, G. Humphrey,
    C. Huttenhower, M. A. Jackson, S. Janssen, D. V. Jeste, L. Jiang, S. T. Kelley,
    D. Knights, T. Kosciolek, J. Ladau, J. Leach, C. Marotz, D. Meleshko, A. V. Melnik,
    J. L. Metcalf, H. Mohimani, E. Montassier, J. A. Navas-Molina, T. T. Nguyen,
    S. Peddada, P. Pevzner, K. S. Pollard, G. Rahnavard, A. Robbins-Pianka,
    N. Sangwan, J. Shorenstein, L. Smarr, S. J. Song, T. Spector, A. D. Swafford,
    V. G. Thackray, L. R. Thompson, Y. Vazquez-Baeza, A. Vrbanac, P. Wischmeyer,
    E. Wolfe, Q. Zhu, The American Gut Consortium, R. Knight.

    Section~\ref{subsection_bloom} has been adapted from the original publication in
    ``Correcting for microbial blooms in fecal samples during room-temperature shipping''.
    \emph{mSystems} A. Amir, D. McDonald, J. A. Navas-Molina, J. Debelius, J. T. Morton,
    E. R. Hyde, A. Robbins-Pianka, R. Knight 2017. DOI: 10.1128/mSystems.00199-16

    Section~\ref{section_memory_exhaustion}, in full, reproduces the material as it
    appears in ``Addressing memory exhaustion failures in Virtual Machines in a cloud environment''.
    J. A. Navas-Molina, S. Mishra. \emph{43rd Annual IEEE/IFIP International Conference on Dependable Systems and Networks (DSN)},
    2013, DOI: 10.1109/DSN.2013.6575330.

    Section~\ref{section_cudswap}, in full, reproduces the material as it
    appears in ``CUDSwap: Tolerating Memory exhaustion failures in cloud computing''.
    J. A. Navas-Molina, S. Mishra. \emph{International Conference on Cloud and Autonomic Computing (ICCAC)},
    2014, DOI: 10.1109/ICCAC.2014.12.

    Section~\ref{section_qiita}, in part, has been submitted for publication of the
    material as it may appear in Nature Methods, 2018, A. Gonzalez, J. A. Navas-Molina,
    T. Kosciolek, D. McDonald, Y. Vazquez-Baeza, S. Janssen, A. D. Swafford, S. B. Orchanian,
    J. G. Sanders, J. Shorenstein, H. Holste, S. Petrus, A. Robbins-Pianka, C. J. Brislawn,
    M. Wang, J. R. Rideout, E. Bolyen, M. Dillon, J. G. Caporaso, P. C. Dorrestein, R. Knight.

    Section~\ref{section_48hours}, in full, reproduces the material as it
    appears in ``From sample to multi-omics conclusions in under 48 hours''.
    R. A. Quinn, J. A. Navas-Molina, E. R. Hyde, S. J. Song, Y. Vazquez-Baeza,
    G. Humphrey, J. Gaffney, J. J. Minich, A. V. Melnik, J. Herschend, J. DeReus,
    A. Durant, R. J. Dutton, M. Khosroheidari, C. Green, R. da Silva, P. C. Dorrestein,
    R. Knight \emph{mSystems}, 2016, DOI: 10.1128/mSystems.00038-16

    Section~\ref{section_platemapper}, in part, has been submitted for publication of the
    material as it may appear in mSystems, 2018, J. A. Navas-Molina, A. Birmingham,
    J. DeReus, J. Sanders, A. Swafford, G. H Humphrey, A. Gonzalez, D. McDonald,
    Y. Vazquez-Baeza, R. Knight.

\end{acknowledgements}


%% VITA
%
%  A brief vita is required in a doctoral thesis. See the OGS
%  Formatting Manual for more information.
%
\begin{vitapage}
\begin{vita}
  \item[2012] B.~S. in Informatics Engineering, Universitat Polit\`ecnia de Catalunya, Barcelona
  \item[2013] M.~Sc. in Computer Science, University of Colorado at Boulder, Boulder
  \item[2018] Ph.~D. in Computer Science, University of California, San Diego
\end{vita}


\begin{publications}

    \item \textsl{Author names marked with $\dagger$ indicate shared first co-authorship}.

    \item \textbf{J. A. Navas-Molina}, E. R. Hyde, J. G. Sanders, R. Knight. ``The microbiome and big data'', \emph{Current Opinion in Systems Biology}, 2017, DOI: 10.1016/j.coisb.2017.07.003.

	\item \textbf{J. A. Navas-Molina}, J. M. Peralta-S\'anchez, A. Gonz\'alez, P. J. McMurdie, Y. V\'azquez-Baeza, Z. Xu, L. K. Ursell, C. Lauber, H. Zhou, S. J. Song, J. Huntley, G. L. Ackermann, D. Berg-Lyons, S. Holmes, J. G. Caporaso, R. Knight. ``Advancing our understanding of the human microbiome using QIIME'', \emph{Methods in Enzymology}, 2013, DOI: 10.1016/B978-0-12-407863-5.00019-8.

	\item E. Kopylova, \textbf{J. A. Navas-Molina}, C. Mercier, Z. Xu, F. Mah\'e, Y. He, H. Zhou, T. Rognes, J. G. Caporaso, R. Knight. ``Open-source sequence clustering methods improve the state of the art'', \emph{mSystems}, 2017, DOI: 10.1128/mSystems.00003-15.

	\item \textbf{J. A. Navas-Molina}, S. Mishra. ``Addressing memory exhaustion failures in Virtual Machines in a cloud environment'', \emph{43rd Annual IEEE/IFIP International Conference on Dependable Systems and Networks}, 2013, DOI: 10.1109/DSN.2013.6575330.

	\item \textbf{J. A. Navas-Molina}, S. Mishra. ``CUDSwap: Tolerating Memory exhaustion failures in cloud computing'', \emph{International Conference on Cloud and Autonomic Computing}, 2014, DOI: 10.1109/ICCAC.2014.12.

	\item A. Amir, D. McDonald, \textbf{J. A. Navas-Molina}, J. Debelius, J. T. Morton, E. R. Hyde, A. Robbins-Pianka, R. Knight. ``Correcting for microbial blooms in fecal samples during room-temperature shipping'', \emph{mSystems}, 2017, DOI: 10.1128/mSystems.00199-16.

	\item A. Amir, D. McDonald, \textbf{J. A. Navas-Molina}, E. Kopylova, J. T. Morton, Z. Xu, E. P. Kightley, L. R. Thompson, E. R. Hyde, A. Gonz\'alez, R. Knight. ``Deblur rapidly resolves single-nucleotide community sequence patterns'', \emph{mSystems}, 2017, DOI: 10.1128/mSystems.00191-16

	\item L. R. Thompson, J. G. Sanders, D. McDonald, A. Amir, J. Ladau, K. J. Locey, R. J. Prill, A. Tripathi, S. M. Gibbons, G. Ackermann, \textbf{J. A. Navas-Molina}, S. Janssen, E. Kopylova, Y. V\'azquez-Baeza, A. Gonz\'alez, J. T. Morton, S. Mirarab, Z. Xu, L. Jiang, M. F. Haroon, J. Kanbar, Q. Zhu, S. J. Song, T. Koscioleck, N. A. Bokulich, J. Lefler, C. J. Brislawn, G. Humphrey, S. W. Owens, J. Hampton-Marcell, D. Berg-Lyons, V. McKenzie, N. Fierer, J. A. Fuhrman, A. Clauset, R. L. Stevens, A. Shade, K. S. Pollard, K. D. Goodwin, J K. Jansson, J. A. Gilbert, R. Knight, The Earth Microbiome Project Consortium. ``A communal catalogue reveals Earth\'s multiscale microbial diversity''. \emph{Nature}, 2017, DOI: http://dx.doi.org/10.1038/nature24621.

	\item $\dagger$R. A Quinn, \textbf{$\dagger$J. A. Navas-Molina}, $\dagger$E. R Hyde, S. Jin Song, Y. V\'azquez-Baeza, G. Humphrey, J. Gaffney, J. J Minich, A. V Melnik, J. Herschend, J. DeReus, A. Durant, R. J Dutton, M. Khosroheidari, C. Green, R. da Silva, P. C Dorrestein, R. Knight ``From sample to Multi-Omics conclusions in under 48 Hours'', \emph{mSystems}, 2016, DOI: 10.1128/mSystems.00038-16.

	\item E. R. Hyde, \textbf{J. A. Navas-Molina}, S. J. Song, J. Kueneman, G. Ackerman, C. Cardona, G. Humphrey, D. Boyer, T. Weaver, J. Mendelson, V. McKenzie, J. Gilbert, R. Knight. ``The oral and skin microbiomes of captive Komodo dragons are significantly shared with their habitat'', \emph{mSystems}, 2016, DOI: 10.1128/mSystems.00046-16.

	\item J. R. Rideout, Y. He, , \textbf{J. A. Navas-Molina}, W. A. Walters, L. K. Ursell, S. M. Gibbons, J. Chase, D. McDonald, A. Gonz\'alez, A. Robbins-Pianka, J. C. Clemente, J. A. Gilber, S. M. Huse, H. W. Zhou, R. Knight, J. C. Caporaso. ``Subsampled open-reference clustering creates consistent, comprehensive OTU definitions and scales to billions of sequences''. \emph{PeerJ}, 2014, DOI: 10.7717/peerj.545.

	\item \noindent\rule[0.5ex]{\linewidth}{0.5pt}

    \textsl{The following publications were not included as part of this dissertation, but were also significant byproducts of my doctoral training.}

	\item Y. V\'azquez-Baeza, A. Gonzalez, L. Smarr, D. McDonald, J. T. Morton, \textbf{J. A. Navas-Molina}, R. Knight. ``Bringing the Dynamic Microbiome to Life with Animations'', \emph{Cell Host and Microbe}, 2017, DOI: 10.1016/j.chom.2016.12.009.

	\item J. T Morton, J. Sanders, R. A Quinn, D. McDonald, A. Gonzalez, Y. V\'azquez-Baeza, \textbf{J. A. Navas-Molina}, S. Jin Song, J. L Metcalf, E. R Hyde, M. Lladser, P. C Dorrestein, R. Knight. ``Balance trees reveal microbial niche differentiation''. \emph{mSystems}, 2017, DOI: 10.1128/mSystems.00162-16.

\end{publications}


\end{vitapage}


%% ABSTRACT
%
%  Doctoral dissertation abstracts should not exceed 350 words.
%   The abstract may continue to a second page if necessary.
%
\begin{abstract}

	Advances in 'omics technologies are producing vast amounts of data, bringing
	microbiome research to a whole new level. This increase in data is pushing the
	limits of existing analysis tools, creating a rapidly-changing environment in which
	new tools are constantly being released. This presents a challenge to
	researchers, who need to constantly learn new analytical tools, expose
	themselves to new environments such as cloud computing or supercomputers,
    and deal with the problems resulting from a heterogeneous environment
	lacking the enforcement of standards. This thesis demonstrates how computational
	optimizations, enforcement of standards, and minimizing the learning curve for
	analytical tools and computational environments empower researchers to push
	the microbiome field forward.

	Chapter~\ref{chapter_overview} motivates and contextualizes the thesis, exposing
	the challenges and opportunities that current microbiome research faces as it
	presents itself as a big data field. Next, Chapter~\ref{chapter_book} presents
	the first gold standard approach for analyzing microbiome data, improvements in analytical tools,
	and examples of how these improvements move microbiome research forward.
	Chapter~\ref{chapter_cudswap} describes a system that lowers the access
	barrier to cloud computing that researchers without a computational background
	face. Chapter~\ref{chapter_qiita} exposes the importance of meta-analyses to
	increase researchers' ability to discover new findings and how much effort
	is currently spent to perform such meta-analyses. This chapter also presents
	Qiita, a web-based system focused on facilitating meta-analyses by enforcing
	standards, normalizing data representation and processing, and providing a
	common interface to current state-of-the-art analysis tools. Chapter~\ref{chapter_rapid_response}
	describes how using the tool improvements and data standardizations presented in
	Chapters~\ref{chapter_book} and~\ref{chapter_qiita}, respectively, and a novel system that
	aids the recording of sample handling information, speed up the process of
	analyzing microbiome samples to levels never reached before. Finally, the concluding
	chapter of this thesis discusses the results and the opportunities opened due
	to these advances, paying special attention to precision medicine, a topic
	in which the microbiome is becoming key.

\end{abstract}


\end{frontmatter}
